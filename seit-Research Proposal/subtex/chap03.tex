\chapter{后续研究及工作计划}
\label{cha:next-step}

\section{主要工作内容}
\label{sec:main-work}
% \begin{figure}[h]
%     \centering
%     \includegraphics[width=\columnwidth]{figures/sec3/main-work}
%     \caption{主要工作内容}
%     \label{fig:main-work}
% \end{figure}

本课题立足于xxxx,围绕xxxx展开研究。基于当前研究现状和技术发展趋势,将研究内容系统性地划分为四个密切相关且递进的研究方向。如图xxx所示,其中三个方向聚焦于xxxx创新,一个方向xxxx优化,形成了完整的技术研究体系:

\subsubsection{xxxxx}

xxxxx

\subsubsection{xxxxx}

xxxx
\subsubsection{xxxxx}

xxxxx
\subsubsection{xxxxx}

作为对前三个研究方向的补充和创新,xxxxx

\section{研究方法、实施计划及其可行性论证}

\subsection{研究方法}
本研究将基于以下核心技术开展,本文的主要研究方法包括但不限于以下内容。

\subsubsection{xxxxx概述}
xxxxx

\subsubsection{xxxxx概述}
xxxxx

\subsubsection{xxxxx概述}
xxxxx


\subsection{实施计划}

本研究的实施计划主要围绕四个核心研究点展开,计划如下:

\subsubsection{研究点1:xxxxx}

\textbf{前期调研:} xxxxx

\textbf{系统建模与算法开发:} xxxxx

\subsubsection{研究点2:xxxxx}

\textbf{前期调研:} xxxxx

\textbf{分布式框架与初步实验:} xxxxx

\subsubsection{研究点3:xxxxx}

xxxxx
\subsubsection{研究点4:xxxxx}

xxxxx

各研究点的实施将按时间进度逐步推进,确保理论探索与实验验证并行开展,以达成预期创新目标。

\subsection{可行性论证}
xxxxx

\section{研究条件及进展}

本课题依托于国家重点研发计划xxxxx

研究点一部分内容已经被接收,xxxxx

\section{研究目标及计划}
\subsection{研究与学位论文预期创新点及目标}

本课题聚焦于6G网络中xxxxx

\subsection{研究与学位论文进度安排}


如表\ref{tab:grad}所示, 工作时间安排分为六个阶段, 涵盖了第一个研究点论文投递工作与后两个研究点的挖掘、研究与实现, 以及毕业论文与毕业答辩的相关准备。

\begin{table}[t]
    \centering
    \caption{毕业设计相关工作拟定时间安排表}
    \renewcommand{\arraystretch}{1.75}
    \begin{tabu}{@{\hspace{0.4cm}}c@{\hspace{0.4cm}}@{\hspace{0.7cm}}>{\centering\arraybackslash}p{7.4cm}@{\hspace{0.4cm}}}
        \tabucline[1.5pt]{-}
        \bfseries{时间}                                     & \bfseries{具体安排}                  \\\tabucline[1.5pt]{-}
        2024 年 11 月 $\sim$ 2024 年 12 月 & 完成第一个研究工作点相关工作论文投递       \\
        2025 年 01 月 $\sim$ 2025 年 06 月 & 开始着手第二、三个工作的相关研究   \\
        2025 年 07 月 $\sim$ 2025 年 12 月 & 完成第二、三个工作相关工作论文投递  \\
        2026 年 01 月 $\sim$ 2026 年 06 月 & 开始着手第四个工作的相关研究       \\
        2026 年 07 月 $\sim$ 2026 年 12 月 & 完成工作相关工作论文投递,完善实验并编写毕业学位论文初稿 \\
        2027 年 01 月 $\sim$ 2027 年 06 月 &  根据盲审意见修改, 准备答辩, 完善终稿
        \\\tabucline[1.5pt]{-}
    \end{tabu}
    \renewcommand{\arraystretch}{1.2}
    \label{tab:grad}
\end{table}